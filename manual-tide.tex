\documentclass[11pt,twoside,a4paper]{article}
%{book}

% This is an automatically generated file.
% Do not edit it.
% Changes to this file are not preserved!

\usepackage{tocloft}
\usepackage{hyperref}
\usepackage{listings}
\lstset{
basicstyle=\small\ttfamily,
columns=flexible,
breaklines=true
}
\setlength{\cftsubsecnumwidth}{3.5em}

\title{Manual for Package:
tide}
\author{Karl Kastner}
%\date{}

\begin{document}

\maketitle

\tableofcontents

% licence
% abstract


\section{@T\_Tide}
\subsection{T\_Tide}
${}$
\begin{lstlisting}
 wrapper for TPXO generated tidal time series

\end{lstlisting}
\subsection{build\_index}
${}$
\begin{lstlisting}
 build a structure whose field names contain the index

\end{lstlisting}
\subsection{from\_tpxo}
${}$
\begin{lstlisting}
 read TPXO output into tidetable object

\end{lstlisting}
\subsection{get\_constituents}
${}$
\begin{lstlisting}
 extract constituents of tpxo object

\end{lstlisting}
\subsection{reorder}
${}$
\begin{lstlisting}
 order constituents as specified by "name"

\end{lstlisting}
\subsection{select}
${}$
\begin{lstlisting}
 select a subsect of constituents

\end{lstlisting}
\subsection{shift\_time\_zone}
${}$
\begin{lstlisting}
 shift phase according to time zone

\end{lstlisting}
\section{@Tidal\_Envelope}
\subsection{Tidal\_Envelope}
${}$
\begin{lstlisting}
 process tidal data to extrac the tidal envelope

\end{lstlisting}
\subsection{init}
${}$
\begin{lstlisting}
 initialize with data

\end{lstlisting}
\section{@Tide\_wft}
\subsection{Tide\_wft}
${}$
\begin{lstlisting}
 wavelet transform of tidal time series

\end{lstlisting}
\subsection{transform}
${}$
\begin{lstlisting}
 wavelet transform tidal time series
 input:
 time   : [1xn] abszissa of input vector, for example time, must be equally spaced
 val    : [1xn] signal, input data series (e.g water level or velocity)
 F      : [1xm] base frequencies, 1, 1, 2, ... for mean level, diurnal, semidirunal ...
 		base periods from base frequencies T=1/F
 n      : [1xm] wavelet window length in multiple of periods
 fc, nc : [scalar] low frequency cutoff and window length in periods
 winstr : [char] fourier windows (kaiser (recommended), hanning, box, etc)
 dt_max : [scalar] maximum time to fill gaps in input data series (recommended 3/24 for tide)
 output:
 tide   : struct with fields
          w_coeff   : [1xn] wavelet coefficients (complex)
          amplitude : amplitude
          phase     : phase
          range     :
          h_tide    :
          h_low     :
          h

\end{lstlisting}
\section{@Tidetable}
\subsection{Tidetable}
${}$
\begin{lstlisting}
 Tide table

\end{lstlisting}
\subsection{analyze}
${}$
\begin{lstlisting}
 extract tidal envelope from time series

\end{lstlisting}
\subsection{export\_csv}
${}$
\begin{lstlisting}
 export tide table to csv file

\end{lstlisting}
\subsection{generate}
${}$
\begin{lstlisting}
 run TPXO to generate time series

\end{lstlisting}
\subsection{generate\_tpxo\_input}
${}$
\begin{lstlisting}
 generate tpxo input table
 Note: superseeded by perl script

\end{lstlisting}
\subsection{import\_tpxo}
${}$
\begin{lstlisting}
 import TPXO data into tidetable object 

\end{lstlisting}
\subsection{plot\_neap\_spring}
${}$
\begin{lstlisting}
 plot average neap and spring tide

\end{lstlisting}
\section{}
\subsection{constituents}
${}$
\begin{lstlisting}

\end{lstlisting}
\subsection{doodson}
${}$
\begin{lstlisting}
 frequency of tidal constituents
 method of doodson
 source: wikipedia

\end{lstlisting}
\subsection{envelope\_amplitude}
${}$
\begin{lstlisting}
 compute envelopes of hw and low water

\end{lstlisting}
\subsection{envelope\_slack\_water}
${}$
\begin{lstlisting}
 slack water envelope of the tide

\end{lstlisting}
\subsection{interval\_extrema}
${}$
\begin{lstlisting}
 times and evelations for high and low water

\end{lstlisting}
\subsection{interval\_extrema2}
${}$
\begin{lstlisting}
 mimimum and maximum within intervals of constant length,
 intended for periodic functions

\end{lstlisting}
\subsection{interval\_zeros}
${}$
\begin{lstlisting}
 times of slack water determined frim velocity u

\end{lstlisting}
\subsection{lunar\_phase}
${}$
\begin{lstlisting}
 lunar phase

\end{lstlisting}
\subsection{rayleigh\_criterion}
${}$
\begin{lstlisting}
 raleigh criterion for resolving tidal constituents
 T > 1/|f1-f2|

\end{lstlisting}
\section{river-tide/@River\_Tide}
\subsection{River\_Tide}
${}$
\begin{lstlisting}
 river tide in a single 1D channel

\end{lstlisting}
\subsection{bcfun}
${}$
\begin{lstlisting}
 boundary conditions

\end{lstlisting}
\subsection{decompose}
${}$
\begin{lstlisting}
 decompose the tide into a right and left travelling wave,
 i.e. into incoming and reflected wave

\end{lstlisting}
\subsection{friction\_coefficient\_dronkers}
${}$
\begin{lstlisting}
 friction coefficient

\end{lstlisting}
\subsection{friction\_coefficient\_godin}
${}$
\begin{lstlisting}
 friction coefficient of Godin

\end{lstlisting}
\subsection{friction\_coefficient\_lorentz}
${}$
\begin{lstlisting}
 Loren't friction coefficient
 c.f. cai
 c.f. dronkers 

\end{lstlisting}
\subsection{friction\_dronkers}
${}$
\begin{lstlisting}
 friction determined by Dronker's method

\end{lstlisting}
\subsection{friction\_exponential\_dronkers}
${}$
\begin{lstlisting}
 friction computed by dronkers method

\end{lstlisting}
\subsection{friction\_godin}
${}$
\begin{lstlisting}
 compute friction by the method of Godin

\end{lstlisting}
\subsection{friction\_trigonometric\_dronkers}
${}$
\begin{lstlisting}
 friction computed by dronkers method
 expressed as Fourier series coefficients
 c.f. dronkers 1964 eq 8.2 and 8.4
 Note: Cai dennominates alpha as phi

\end{lstlisting}
\subsection{friction\_trigonometric\_godin}
${}$
\begin{lstlisting}
 friction computed by the method of godin
 expressed in fourier series coefficients

\end{lstlisting}
\subsection{friction\_trigonometric\_lorentz}
${}$
\begin{lstlisting}
 friction computed by the method of Lorent'z
 expressed as Fourier series coefficients

\end{lstlisting}
\subsection{init}
${}$
\begin{lstlisting}
 solve backwater equation for surface level
 TODO this should not be solved as a ivp but included in the bvp iteration
 TODO generate the mesh here and precompute fixed values instead of passing functions

\end{lstlisting}
\subsection{mwl\_offset}
${}$
\begin{lstlisting}
 offset of the tidally averaged surface elevation caused by tidal friction
 Linear estimate of the mean water level offset (ignoring feed-back of tide)

\end{lstlisting}
\subsection{odefun}
${}$
\begin{lstlisting}
 coefficients of the backwater and wave equation for river-tides

\end{lstlisting}
\subsection{odefun0}
${}$
\begin{lstlisting}
 coefficients of the backwater equation for the river tide

\end{lstlisting}
\subsection{odefun1}
${}$
\begin{lstlisting}
 differential equation coefficients of the main tidal species
 f1 Q'' + f2 Q' + f3 Q + f4 = 0

\end{lstlisting}
\subsection{odefun2}
${}$
\begin{lstlisting}
 ordinary differential quation coefficients of the even overtide

\end{lstlisting}
\subsection{q2z}
${}$
\begin{lstlisting}
 tidal component of surface elevation determined from tidal discharge

 by continuity

 %   dz/dt + dq/dx = 0
 => i o z = - dq/dx
 =>     z = -1/(io) dq/dx
 =>     z = 1i/o dq/dx

 TODO allow Q as input

\end{lstlisting}
\subsection{solve}
${}$
\begin{lstlisting}
 call stationary or non-stationary solver respectively

\end{lstlisting}
\subsection{solve\_swe}
${}$
\begin{lstlisting}
 determine river tide by the fully non-stationary FVM and then extract the tide

\end{lstlisting}
\subsection{solve\_wave}
${}$
\begin{lstlisting}
 solve for the oscillatory (tidal) componets

\end{lstlisting}
\subsection{wave\_number\_analytic}
${}$
\begin{lstlisting}
 analytic expression of the wave number

 valid for both tidally, river dominated and low friction conditions
 and converging channels

 k      : complex wave number in a reach with constant width and bed slope
 im(k)  : damping modulus (rate of amplitude change)
 re(k)  : actual wave number (rate of phase change)

 c.f. derive_wave_number


\end{lstlisting}
\subsection{wave\_number\_approximation}
${}$
\begin{lstlisting}
 approximate wave number of the left and right traveling wave for variable coefficients

\end{lstlisting}
\section{river-tide/@River\_Tide\_Cai}
\subsection{Gamma}
${}$
\begin{lstlisting}
 Gamma parameter for tidal propagation
 c.f. Cai 2014

\end{lstlisting}
\subsection{River\_Tide\_Cai}
${}$
\begin{lstlisting}
 prediction of river tide by the method of Cai (2014)

\end{lstlisting}
\subsection{river\_tide\_cai\_}
${}$
\begin{lstlisting}
 determine the surface amplitude of the river tide
 c.f. Cai

\end{lstlisting}
\subsection{rt\_quantities}
${}$
\begin{lstlisting}
 determine the quantities that determine the tidal propagation
 c.f. Cai
 
 Note: this computes 4 unknowns following Cai, however,
 lambda, mu and epsilon can be substituted
 making it an equation in one unknown (delta) only

\end{lstlisting}
\section{river-tide/@River\_Tide\_Empirical}
\subsection{River\_Tide\_Empirical}
${}$
\begin{lstlisting}
 class for fitting models to at-a-station time series of tidal elevation

\end{lstlisting}
\subsection{fit\_amplitude}
${}$
\begin{lstlisting}
 fit the oscillatory components

\end{lstlisting}
\subsection{fit\_mwl}
${}$
\begin{lstlisting}
 fit the tidally averaged water level

\end{lstlisting}
\subsection{fit\_phase}
${}$
\begin{lstlisting}
 fit the phase of the oscillatory components

\end{lstlisting}
\subsection{fit\_range}
${}$
\begin{lstlisting}
 fit the tidal range

\end{lstlisting}
\subsection{predict\_amplitude}
${}$
\begin{lstlisting}
 predict the oscillatory components

\end{lstlisting}
\subsection{predict\_mwl}
${}$
\begin{lstlisting}
 predict the mean water level

\end{lstlisting}
\subsection{predict\_phase}
${}$
\begin{lstlisting}
 predict tidal phase 

\end{lstlisting}
\subsection{predict\_range}
${}$
\begin{lstlisting}
 predict the tidal range

\end{lstlisting}
\subsection{rt\_model}
${}$
\begin{lstlisting}
 select the model for fitting

\end{lstlisting}
\section{river-tide/@River\_Tide\_Map}
\subsection{River\_Tide\_Map}
${}$
\begin{lstlisting}
 container class to store individual river tide scenarios

\end{lstlisting}
\subsection{d2au1\_dx2}
${}$
\begin{lstlisting}
 second derivative of the tidal velocity magnitude

 note: this is for finding zeros,
       the true derivative has to be scaled  up by z

\end{lstlisting}
\subsection{d2az1\_dx2}
${}$
\begin{lstlisting}
 second derivative of the tidal surface elevation

 note: this is for finding zeros,
       the true derivative has to be scaled  up by z

\end{lstlisting}
\subsection{dkq\_dx}
${}$
\begin{lstlisting}
 along-channel derivative of the wave number of the discharge
 neglects width variation

 TODO, rederive with g as variable

\end{lstlisting}
\subsection{dkz\_dx}
${}$
\begin{lstlisting}
 along channel derivative of the wave number of the tidal surface elevation
 ignores width variation dh/dx and second order depth variation (d^2h/dx^2)
 TODO rederive with g symbolic

\end{lstlisting}
\subsection{fun}
${}$
\begin{lstlisting}
 compute a specific river tide scenario and store it in the hash,
 or retrive the scenario, if it was already computed

\end{lstlisting}
\subsection{key}
${}$
\begin{lstlisting}
 key for storing a scenario

\end{lstlisting}
\subsection{plot}
${}$
\begin{lstlisting}
 plot result

\end{lstlisting}
\section{river-tide/@Tidal\_River\_Network}
\subsection{Tidal\_River\_Network}
${}$
\begin{lstlisting}
 tide in a fluvial delta channel network, extension of 1D river tide
 the network is a directed graph

\end{lstlisting}
\subsection{discharge\_amplitude}
${}$
\begin{lstlisting}
 discharge amplitude

\end{lstlisting}
\subsection{mean\_water\_level}
${}$
\begin{lstlisting}
 predict the mean water level

\end{lstlisting}
\subsection{plot\_mean\_water\_level}
${}$
\begin{lstlisting}
 plot tidally averaged water level

\end{lstlisting}
\subsection{plot\_water\_level\_amplitude}
${}$
\begin{lstlisting}
 plot surface elevation amplitude

\end{lstlisting}
\subsection{solve}
${}$
\begin{lstlisting}
 solve for the tide in a fluvial chanel network

 boundary condition at end points not connected to junctions
 	[ channel 1 id, endpoint id (1 or 2), s0, c0
        ...
         channel n id, endpoint id (1 or 2), s0, c0]

 conditions at junctions are specified as cells
	 each cell contains an nx2 array
	 n : number of connecting channels
	 [channel id1, endpoint id (1 or 2), ...
         channel idn, endpoint id (1 or 2)]

 every tidal species for each channel has 4 unknowns
 these are 2x2 unknowns for the sin + cos of left and right going wave

\end{lstlisting}
\subsection{water\_level\_amplitude}
${}$
\begin{lstlisting}
 predict the surface elevation amplitude

\end{lstlisting}
\section{river-tide/River\_Tide\_JK}
\subsection{River\_Tide\_JK}
${}$
\begin{lstlisting}

\end{lstlisting}
\subsection{jk\_damping\_modulus}
${}$
\begin{lstlisting}
 damping modulus of the river tide
 c.f. Jay and Kukula

\end{lstlisting}
\subsection{jk\_mean\_level}
${}$
\begin{lstlisting}
 tidally averaged surface elevation
 c.f. Jay and Kukulka

\end{lstlisting}
\subsection{jk\_rivertide\_predict}
${}$
\begin{lstlisting}
 predict river tide by the method of jay and kukulka

\end{lstlisting}
\subsection{jk\_rivertide\_regress}
${}$
\begin{lstlisting}
 Regression of tidal coefficients according to Jay & Kulkulka

 coefficients of the r-regression factor 2 apart for specis (jay C7)
 this can be repeated for each tidal species (diurnal, semidiurnal)

\end{lstlisting}
\subsection{jk\_tidal\_discharge}
${}$
\begin{lstlisting}
 tidal discharge
 c.f. Jay and Kukulka

\end{lstlisting}
\subsection{jk\_tidal\_range}
${}$
\begin{lstlisting}
 predict tidal range

\end{lstlisting}
\section{river-tide}
analysis and prediction of river tides
 

\subsection{damped\_wave\_bvp}
${}$
\begin{lstlisting}
 analytic solution to the river tide formulated as boundary value problem
 in a river with finite length

 c.f. Godin 1986


\end{lstlisting}
\subsection{damped\_wave\_ivp}
${}$
\begin{lstlisting}
 linearly damped wave in rectangular channel
 x_t = Ax + b

\end{lstlisting}
\subsection{damping\_modulus\_river}
${}$
\begin{lstlisting}
 damping modulus of the tidal wave for river flow only

\end{lstlisting}
\subsection{damping\_modulus\_tide}
${}$
\begin{lstlisting}
 damping modulus of the tide without river flow
 c.f. friedrichs, ippen harleman
 output :
 k : wave number
 re(k) : rate of phase change
 im(k) : damping rate

\end{lstlisting}
\subsection{rdamping\_to\_cdrag\_tide}
${}$
\begin{lstlisting}
 converts damping rate to drag coefficient
 c.f. friedrichs, ippen harleman

\end{lstlisting}
\subsection{rt\_celerity}
${}$
\begin{lstlisting}
 celerity of the tidal wave

\end{lstlisting}
\subsection{rt\_quasi\_stationary\_complex}
${}$
\begin{lstlisting}
 quasi-stationary solution of the SWE
 TODO staggered grid does not help: q1' needed

\end{lstlisting}
\subsection{rt\_quasi\_stationary\_trigonometric}
${}$
\begin{lstlisting}
 quasi statinary form of the SWE

\end{lstlisting}
\subsection{rt\_reflection\_coefficient\_gradual}
${}$
\begin{lstlisting}
 reflection coefficient for gradual varying cross section geometry
 without damping

\end{lstlisting}
\subsection{rt\_wave\_equation}
${}$
\begin{lstlisting}
 solve river tide as boundary value problem

 input:
 omega : [nfx1] angluar frequency of tidal component, zero for mean flow
 reach : [nrx1] struct
 .L    : [1x1]  length of reaches
	 .width(x,h)	width
	 .bed(x,h)	bed level
	 .surface(x,h)	surface elevation
	 .Cd(x,h)	drag coefficient
 .bc   : [nd,nf] boundary/junction conditions
	  bc(id,if).type : {surface, velocity, discharge} (dirichlet)
	  bc(id,if).val  : value
 opt   : [1x1] struct
	- constant surface elevation
	- deactivative advective acceleration
 	.dx : spatial resolution

 dimensions:
	nr : nurmber or reaches
	nd : upstream/downstream index
	nf : frequency index

\end{lstlisting}
\subsection{rt\_z2q}
${}$
\begin{lstlisting}
 determine tidal discharge from water level for tidal wave

\end{lstlisting}
\subsection{test\_rt\_wave\_number}
${}$
\begin{lstlisting}

\end{lstlisting}
\subsection{tidal\_ellipse}
${}$
\begin{lstlisting}
 tidal ellipse, numerical ode solution

\end{lstlisting}
\subsection{wavetrainz}
${}$
\begin{lstlisting}
 determine river tide by iterated integration of the surface elevation

\end{lstlisting}
\subsection{wavetwopassz}
${}$
\begin{lstlisting}
 two pass solution for the linearised wave equation, for surface elevation

\end{lstlisting}
\section{test}
\subsection{test\_tidal\_harmonic\_analysis}
${}$
\begin{lstlisting}

\end{lstlisting}
\section{}
\subsection{tidal\_constituents}
${}$
\begin{lstlisting}

\end{lstlisting}
\subsection{tidal\_energy\_transport\_1d}
${}$
\begin{lstlisting}
 energy transport of a tidal wave

\end{lstlisting}
\subsection{tidal\_envelope}
${}$
\begin{lstlisting}
 envelope of the tide

 input : t time in days
         f surface elevation
 ouput : tl time of low water
         vl surface elevation at low water
         ldx index of low water
         th time of high water
         vh surface elevation at high water
         hdx index of high water
         ndx neap index
         sdx spring index
         dmax:
         drange: range per day

\end{lstlisting}
\subsection{tidal\_envelope2}
${}$
\begin{lstlisting}
 surface levelation envelope of the tide
 low water, high water and tidal range for lunar each day
 

 input:
        time  : 
        L     : surface elevation
        order : interpolation order (default 2)
 ouput:
        timei : vector eqispaced
        lmini : minimum level
        lmaxi : maximum level
        rangei : range
        midrangei : (min + max)/2, usually different from mean
        phii  : pseudo phase

 Note: the pseudo phase phi jumps, this is because if the tide is semidiurnal,
       sometimes the lower hw becomes the next day higher then than the 
	current high water, e.g. there is no smooth transition by
       51min but a jump by 12h

\end{lstlisting}
\subsection{tidal\_harmonic\_analysis}
${}$
\begin{lstlisting}
 tidal_harmonic analysis

\end{lstlisting}
\section{tide-savenije}
\subsection{savenije\_phase\_lag}
${}$
\begin{lstlisting}
 phase lag of high and low water

 phi : u_river/u_tide < 1

 delta_eps_hw = omega*(t_hws - t_hw)
 delta_eps_hw = omega*(t_lws - t_lw)

 c.f. savenije

\end{lstlisting}
\subsection{savenije\_tidal\_range}
${}$
\begin{lstlisting}
 tidal range

 based on Savenije 2012

 x    : distance to river mouth
 eta  : range
 eta0 : range at river mouth
 hbar : mean water depth
 phi  : velocity ratio u_tide/u_river 
	 note: this varies in strongly convergent estuaries
 K    : mannings coefficient
 I    : residual surface slope I

\end{lstlisting}
\subsection{savenije\_tidal\_range1}
${}$
\begin{lstlisting}
 tidal range

 based on Horrevoets/Savenije, 2004

 H0   : tidal range at river mouth
 h0   : initial water depth
 v    : velocity scale
 b    : convergence length
 sine : phase lag
 K    : Mannings coefficient
 Q_r   : river discharge

\end{lstlisting}
\subsection{savenije\_timing\_hw\_lw}
${}$
\begin{lstlisting}
 time of high water and low water
 c.f. savenije 2012

\end{lstlisting}
\subsection{tide-savenije}
${}$
\begin{lstlisting}

\end{lstlisting}
\end{document}
